
\subsection{Object3D}

Object3Dはシーンの構成要素(シーンオブジェクト)の基底となる抽象クラスです。
全てのシーンの構成要素はこのクラスを継承します。

\subsubsection{アニメーション}

addAnimationTrack()メソッドでこのオブジェクトにアニメーションを追加します。

\begin{verbatim}
    virtual void addAnimationTrack (AnimationTrack* animation_track);
\end{verbatim}

引数には適切に設定されたAnimationTrackオブジェクトを渡します。
アニメーショントラックのターゲットプロパティが、
このオブジェクトに適応可能でなければいけません。

既に設定したアニメーションを取り除くにはremoveAnimationTrack()メソッドを使います。

\begin{verbatim}
    void removeAnimationTrack (AnimationTrack* animation_track);
\end{verbatim}

引数には取り除きたいAnimationTrackオブジェクトを指定します。

\subsubsection{ユーザーID}

全てのシーンオブジェクトは任意のユーザーIDを持ちます。
ユーザーIDの設定および解釈はライブラリの使用者の自由です。
ユーザーIDの設定にはsetUserID()メソッドを使います。
デフォルトは0です。

\begin{verbatim}
    void setUserID (int userID);
\end{verbatim}

現在設定されているのユーザーIDを取得するにはgetUerID()メソッドを使います。

\begin{verbatim}
    int getUserID () const;
\end{verbatim}

\subsubsection{ユーザーオブジェクト}

全てのシーンオブジェクトは任意のユーザーオブジェクトを1つ持ちます。
ユーザーオブジェクトはキー、バリューからなる任意のデータで、
その設定と解釈はライブラリの使用者の自由です。
この機能は現在実装されていません。

ユーザーオブジェクトを設定するにはsetUserObject()メソッドを使います。

\begin{verbatim}
    void setUserObject (const char* name, void* value);
\end{verbatim}

第1引数にはキーとなる文字列を、第2引数には値となるデータを渡します。

現在設定されているユーザーオブジェクトを取得するにはgetUserObject()メソッドを使います。

\begin{verbatim}
    void* getUserObject () const;
\end{verbatim}


\subsubsection{デバッグ表示}

M3G非標準ですがシーンの構成要素を表すクラスは全てデバッグ用にのprint()メソッドを持ちます。
print()メソッドはそのクラスの情報のみを表示するので、基底クラスの情報も表示したい場合は
明示的に基底クラスの名前でprint()メソッドを修飾して呼び出してください。
例えばMeshクラスで全ての情報を表示するには以下のように書きます。

\begin{verbatim}
mesh->print (cout);
mesh->Node:: print (cout);
mesh->Transformable:: print (cout);
mesh->Object3D:: print (cout);
\end{verbatim}




\begin{verbatim}
\end{verbatim}




