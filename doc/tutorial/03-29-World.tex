\subsection{ワールド}

Worldクラスはシーンを構成する最上位のコンテナクラスです。
Worldは他のノードの子になる事ができません。
シーンはWorldオブジェクトをルートとする木構造です。

WorldクラスをGraphics3Dクラスのrender()メソッドに渡す事で、
そのシーンがレンダリングされます。

\begin{verbatim}
World* wld = new World;
g3d->render (wld);
\end{verbatim}

\subsubsection{カメラの設定}

レンダリングするためにはsetActiveCamera()メソッドで
カメラを1つアクティブ化する必要があります。

\begin{verbatim}
    void setActiveCamera (Camera* camera);
\end{verbatim}

カメラは事前にシーンノードとしてWorldに挿入されている必要があります。
詳細はCameraクラスの説明をご覧ください。

\subsubsection{背景の設定}

背景の設定はBackgroundクラスが行います。
デフォルト以外の背景を使う場合はsetBackground()メソッドで設定します。
詳細はBackgroundクラスの説明をご覧ください。

\begin{verbatim}
    void setBackground (Background* background);
\end{verbatim}


\subsubsection{ノードの追加}

シーンノードの追加にはaddChild()メソッドを使います。
ただしaddChild()はWorldクラスの基底クラスであるGroupクラスで定義されたメソッドです。
詳細はGroupクラスの説明をご覧ください。

\begin{verbatim}
    void addChild (Node* child);
\end{verbatim}






